% How the problem is analysed to create the solution.
% Overall architecture of the design.
% Examine the design approaches taken.
% Identification of areas of the design that account for the requirements and resolve potential conflicts. 
% 

\chapter{Architecture and Algorithms} % (fold)
\label{cha:architecture_and_algorithms}
% This section will contain:
% \begin{itemize}
%     \item \textbf{Architectural Overview:} An overview of the system architecture, linked to the high-level requirements.
%     This will touch on the portions of the system that were ruled out of scope later on, and why.
%     \item \textbf{System Component Examination:} An examination of the design of each of the system components in detail.
%     \item \textbf{The Development of Metaspec:} The process of developing the metaspec language itself, as well as some syntax examples and full syntax description. 
%     \item \textbf{Algorithmic Descriptions:} The verification engine in particular is very heavy on algorithmic work. 
%     These algorithms will be explained here, accompanied by any additional proof work that they require. 
%     This will include proofs for the special form semantics.
%     It will also look at initial or discarded algorithmic designs as part of the process. 
% \end{itemize}

As a project, \gls{absol} has had a very heavy research bent. 
The experimental nature of the toolchain resulted in a heavy up-front design load and, combined with the highly theoretical nature of the language verification algorithms, this meant that design and algorithmic development dominated the time spent on the project.\\

This section aims to illustrate the significant design work that was put into all facets of the project.
It first explores the design process for the metalanguage, \gls{metaspec}, demonstrating the careful thought that went into the final result. 
It then provides a high-level overview of the architecture of the toolchain, showing the main system components and tying them to the overarching system requirements.
Finally, it concludes with the design of both the core algorithms and special language features that are integral to the operation of \gls{absol}. 

\section{Designing the Metalanguage --- Metaspec} % (fold)
\label{sec:designing_the_metalanguage_metaspec}
% Careful consideration of why each bit of syntax looks like it does, what it allows the user to do. 
% Design for intuition and flexibility.
% Look like the source kind of semantics -> provide examples. 
% Consideration of where the semantics needed to be placed in the syntax for it to make real sense. 

\gls{metaspec} is the metalanguage for the \gls{absol} project, allowing the language designers to specify both the syntax and semantics of their DSL, as well as associated metadata, in a unified form. 
The final syntax for Metaspec is the result of significant design work, and consequentially the syntax discussed below has been through some evolution. \\

The complete grammar for Metaspec can be found in Appendix~\ref{cha:the_metaspec_grammar}, and is written in standard \gls{ebnf} notation. 
The same notation will be used throughout this section of the document. 

\subsection{The Top-Level Definitions} % (fold)
\label{sub:the_top_level_definitions}

% subsection the_top_level_definitions (end)

\subsection{Specifying the Language Syntax} % (fold)
\label{sub:specifying_the_language_syntax}
% Why were the changes from EBNF made? 
% Removal of empty syntax

% subsection specifying_the_language_syntax (end)

\subsection{Specifying the Language Semantics} % (fold)
\label{sub:specifying_the_language_semantics}
% Semantics and the unspecified cases. 

% subsection specifying_the_language_semantics (end)

\subsection{Combining Syntax and Semantics} % (fold)
\label{sub:combining_syntax_and_semantics}
% Challenges choosing the combination point
% Initially at the production level, but this was nonsensical
% What does having it at the alternation level allow users to do?
% Why the particular form was chosen: --> {};

% subsection combining_syntax_and_semantics (end)

% section designing_the_metalanguage_metaspec (end)

\section{Designing the Metacompiler --- ABSOL} % (fold)
\label{sec:designing_the_metacompiler_absol}
% TODO Parser: Lexer, Parser, Etc.
% TODO Verifier: Different Verification Modules, talk about the recursive nature. Preprocessor.
% Use this same style of architectural overview in the development section. 
% Architectural overview focuses on system components, not code-level components.

As for any large system, it is important to be able to visualise the way in which the individual system components interact and are integrated. \\

\gls{absol}, the metacompiler system 

\subsection{Lexing and Parsing} % (fold)
\label{sub:lexing_and_parsing}
% Talk about the design for the lexer and parser, informed by the use of Megaparsec.
% Talk about the process and use of datatypes to represent a very strongly typed AST

% subsection lexing_and_parsing (end)

\subsection{The Verification Engine} % (fold)
\label{sub:the_verification_engine}
% Talk about the different modules, the verification preprocessor, the recursive nature of the algorithm for traversing the metaspec AST.

% subsection the_verification_engine (end)

% section designing_the_metacompiler_absol (end)

\section{The Core Algorithms} % (fold)
\label{sec:the_core_algorithms}
% Talk about the design of each of the core algorithms in detail.

\subsection{Verifier Traversal} % (fold)
\label{sub:verifier_traversal}

% subsection verifier_traversal (end)

\subsection{Semantic Form Verification} % (fold)
\label{sub:semantic_form_verification}
% All three criteria (evals, subterms, etc)

% subsection semantic_form_verification (end)

\subsection{Guard Checking} % (fold)
\label{sub:guard_checking}

% subsection guard_checking (end)

% section the_core_algorithms (end)

\section{Special Language Features} % (fold)
\label{sec:special_language_features}
% Talk about the design of each of the language features, and prove the required termination properties here.

\subsection{Feature --- \texttt{base}} % (fold)
\label{sub:feature_base}

% subsection feature_base (end)

\subsection{Feature --- \texttt{number}} % (fold)
\label{sub:feature_number}

% subsection feature_number (end)

\subsection{Feature --- \texttt{string}} % (fold)
\label{sub:feature_string}

% subsection feature_string (end)

\subsection{Feature --- \texttt{list}} % (fold)
\label{sub:feature_list}

% subsection feature_list (end)

\subsection{Feature --- \texttt{matrix}} % (fold)
\label{sub:feature_matrix}

% subsection feature_matrix (end)

\subsection{Feature --- \texttt{traverse}} % (fold)
\label{sub:feature_traverse}

% subsection feature_traverse (end)

\subsection{Feature --- \texttt{funcall}} % (fold)
\label{sub:feature_funcall}

% subsection feature_funcall (end)

% section special_language_features (end)

% chapter architecture_and_algorithms (end)
