\chapter{Evaluation} % (fold)
\label{cha:evaluation}
This section will include:
\begin{itemize}
    \item An examination of how the system meets it high-level goals.
    \item Identification and discussion of areas requiring improvement, alteration and further work.
    \item Concluding with discussion of the major successes and failures of the ABSOL system.
\end{itemize}

% Flaws of the metalanguage:
%   - No direct use of environment accesses in basic semantics, requiring indirection.
%   - Fairly awkward use of the environment (no ability to delete properties, for example). 
%   - Grammar admits some nonsensical terms such as alternations with semantics not at the top-level. These are ignored by the metacompiler, but are inelegant. Ideally this would not be allowed by the grammar at all.
% Environment not particularly well thought-out.

% Could have used a FAR more rigorous design process and treated it more like a traditional software project.

% - Funcall and the other special forms are important for the quality of the project.
% - Discuss what would've otherwise been done in 'future' work -> Show that it is still possible, and indicate how the drive in this direction has been influencing the existing portions of the project. 
% Error reporting could be far better in some circumstances - more accurate diagnostics would be a great help to the language designer. 

% Contributions
% Failures
% Future Work

% chapter evaluation (end)
