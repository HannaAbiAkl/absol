\chapter{Software Readme} % (fold)
\label{cha:software_readme}
This section will contain a guide to the project as a whole, as well as the necessary details to write languages and run the metacompiler on them.

\section{The Project Structure} % (fold)
\label{sec:the_project_structure}
The implementations for each of the metacompiler components are contained in the following files. 
Each of these files is given relative to the project root in \mintinline{text}{absol/}:
\begin{itemize}
    \item \textbf{The Metacompiler Front-End:} The files for the front-end are contained in:
    \begin{itemize}
        \item \mintinline{text}{./app/Main.hs}
        \item \mintinline{text}{./app/Cmdargs.hs}
    \end{itemize}
    \item \textbf{Metaparse:} The files for metaparse are contained in:
    \begin{itemize}
        \item \mintinline{text}{./src/Absol/Metaparse.hs}
        \item \mintinline{text}{./src/Absol/Metaparse/Grammar.hs}
        \item \mintinline{text}{./src/Absol/Metaparse/Parser.hs}
        \item \mintinline{text}{./src/Absol/Metaparse/Utilities.hs}
        \item \mintinline{text}{./src/Absol/Metalex.hs}
        \item \mintinline{text}{./src/Absol/Metaspec.hs}
        \item \mintinline{text}{./src/Absol/Metaspec/Special.hs}
    \end{itemize}
    \item \textbf{Metaverify:} 
    \begin{itemize}
        \item \mintinline{text}{./src/Absol/Metaverify.hs}
        \item \mintinline{text}{./src/Absol/Metaverify/Collate.hs}
        \item \mintinline{text}{./src/Absol/Metaverify/Diagnostics.hs}
        \item \mintinline{text}{./src/Absol/Metaverify/RuleTag.hs}
        \item \mintinline{text}{./src/Absol/Metaverify/State.hs}
    \end{itemize}
    \item \textbf{Misc:} There are also stub implementation files for other components of the metacompiler toolchain that have since been declared as out of the project scope. 
    \item \textbf{Metaspec Examples:} These consist of the file used for testing the metacompiler and the sample implementation of `spreadsheet' (see \autoref{sec:an_example_language}).
    They can be found in the \mintinline{text}{./metaspec/} directory.
\end{itemize}

% section the_project_structure (end)

\section{Building and Executing ABSOL} % (fold)
\label{sec:building_and_executing_absol}
To build and execute the metacompiler please follow the instructions in the following sections.

\subsection{System Requirements} % (fold)
\label{sub:system_requirements}
In order to both build and execute \gls{absol} you will need to have the following software available:
\begin{itemize}
    \item \textbf{Stack:} The Haskell build management tool is able to handle the downloading and installing of both the required compiler version and required libraries.
    \item \textbf{Internet Connection:} Stack requires access to the internet to download and build the packages that are depended upon by \gls{absol}. 
\end{itemize}

% subsection system_requirements (end)

\subsection{Building the Metacompiler} % (fold)
\label{sub:building_the_metacompiler}
Building \gls{absol} is a simple prospect as long as the system requirements are met.
It can be done as follows:
\begin{enumerate}
    \item Open a terminal.
    \item Navigate to the root directory of the project as described in \autoref{sec:the_project_structure}.
    \item Execute the command \mintinline{text}{stack setup}. This will download and install the appropriate versions of GHC and its support libraries. 
    \item Execute the command \mintinline{text}{stack build}. 
\end{enumerate}

This will result in a binary being built for your particular system configuration.
The binary can then be executed as described in \autoref{sub:executing_the_metacompiler}.

% subsection building_the_metacompiler (end)

\subsection{Executing the Metacompiler} % (fold)
\label{sub:executing_the_metacompiler}
Once the metacompiler has been built, it can be executed on a provided test file or one of your own creation.
Test files are provided in the \mintinline{text}{./metaspec/} directory. 
One of these is the example language developed in \autoref{sec:an_example_language}, while the other is the test file discussed in \autoref{sec:testing_during_development}.
The metacompiler can be executed as follows (assuming that you have a terminal open in the same directory as described above):
\begin{enumerate}
    \item Choose a metaspec file and note its relative path to your current directory. 
    This is denoted \mintinline{text}{FILE} in the following command.
    \item First run \mintinline{text}{stack exec absol -- --help}. 
    This will display the help text for all of the command-line options.
    \item Next run \mintinline{text}{stack exec absol -- FILE -l}.
    This will execute the metacompiler on \mintinline{text}{FILE} in verbose reporting mode. 
\end{enumerate}

If you have used one of the example files you will see that the metacompiler is able to verify the file such that it terminates.
Please feel free to introduce errors into either of these files to examine how the metacompiler behaves in the presence of errors. 
If you continue to include the \mintinline{text}{-l} flag in these circumstances you will see comprehensive diagnostic information about whatever error you introduce. 

% subsection executing_the_metacompiler (end)

% section building_and_executing_absol (end)

% chapter software_readme (end)
